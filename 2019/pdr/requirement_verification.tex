\begin{longtable}{@{\makebox[0.08\textwidth][l]{\rownumber}} | p{0.5\textwidth} | p{0.42\textwidth}}
	\textbf{Requirement} & \textbf{Verification} 
	\gdef\rownumber{Req.\space\stepcounter{magicrownumbers}\arabic{magicrownumbers}} \\ \hline
	MTOW of $6.9kg$. & Weight budget supported by component and final assembly measurements.\\ \hline
	Maximum 4 cell LiPo battery. & Voltage measurement. \\ \hline
	Payload: Single First Aid Kit, one or more Buxton water bottles. & CAD of fuselage in combination with exact dimensions of payload, flight tests. \\ \hline
	Completely autonomous operation. & Hands-off, autonomous flight testing of all mission scenarios in addition to hardware- and software-in-the-loop simulations. \\ \hline
	Take off and landing within $30\si{m}$ & Basic performance calculations and tests. \\ \hline
	All radio equipment must be licensed for use in the UK and have a minimum range of 1km. Radio equipment providing control of the UAS and for the FTS must be 'spread spectrum', and must operate on the 2.4GHz band. & See Section 5.2, RF Compliance. \\ \hline
	The UAV must have a FTS which is either activated 5s after the uplink is lost or manually by the flight safety officer via the master controller, and 10s after the downlink is lost. The FTS will also be activated in the case of a geo fence breach. & See Section 5.3, Flight Termination System. \\ \hline
	The UAS should carry a camera system with target recognition capability to undertake target search. & Raspberry Pi with gimbal-mounted camera on UAV. Recognition capability tested in simulation and in flight. \\ \hline
	The following telemetry must be available in flight: UAS position on moving map, local airspace, QFE, IAS. & Positional data relayed via telemetry to base-station with moving map on GUI. \\ \hline
	The aircraft must allow for the fitting of a WBT-201 "G-Rays 2" GPS Tracker. & CAD of fuselage, using dimensions of tracker provided in rules document. \\ \hline
	Batteries must be coloured brightly. & Visual inspection of batteries. \\ \hline
	Carrying and dropping of payload on demand. & CAD of fuselage, flight tests.\\ \hline
	The UAV should be as accurate as possible when delivering the payloads. & Usage of flaps for slow flying during payload drop. \\ \hline
	Payload must remain intact during impact & Parachutes as speed retardation systems. \\ \hline
	The aircraft should carry as much payload mass as possible. & Use of composite materials (CFRP, GFRP) for lightweight airframe design. \\ \hline
	The UAV should navigate as accurately as possible. & Flight test via pre-defined GPS waypoints, error measurement between actual flight path and ideal trajectory. \\ \hline
	The mission should be completed in under 10 min. & Performance calculations factoring in wind and expected distance covered, flight tests. \\
\end{longtable}

